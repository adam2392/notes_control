\documentclass[class=article, crop=false]{standalone}
\usepackage[utf8]{inputenc} % allow utf-8 input
\usepackage[T1]{fontenc}    % use 8-bit T1 fonts
\usepackage{url}            % simple URL typesetting
\usepackage{booktabs}       % professional-quality tables
\usepackage{amsfonts}       % blackboard math symbols
\usepackage{nicefrac}       % compact symbols for 1/2, etc.
\usepackage{microtype}      % microtypography
\usepackage{lipsum}
\usepackage{amsmath}
\usepackage{amsthm}
\usepackage{hyperref}
\usepackage{import}
\usepackage[subpreambles=true]{standalone}
\hypersetup{
    colorlinks=true, %set true if you want colored links
    linktoc=all,     %set to all if you want both sections and subsections linked
    linkcolor=blue,  %choose some color if you want links to stand out
}

% \theoremstyle{definition}
% \newtheorem{definition}{Definition}[section]

% \theoremstyle{remark}
% \newtheorem*{remark}{Remark}

% \theoremstyle{lemma}
% \newtheorem*{lemma}{Lemma}

% \theoremstyle{theorem}
% \newtheorem*{theorem}{Theorem}

% \theoremstyle{corollary}
% \newtheorem*{corollary}{Corollary}

% \theoremstyle{property}
% \newtheorem*{property}{Property}
% \usepackage[subpreambles=true]{standalone}
% \usepackage{import}
\begin{document}

\section{Important Facts}
	\subsection{Finite-Dimensional Vector Spaces}
	We begin by defining \textbf{finite-dimensional} vector spaces, which are fundamental for matrix analysis. A vector space, V over a field, F, is a set of objects called vectors that have the following properties:

	i) closed under binary addition
	ii) associative: $x$
	iii) commutative
	iv) has the zero vector

	A field is typically the real numbers, or complex numbers, and has the following properties:

	\begin{enumerate}
		\item Closure under addition: $\forall x, y \in F$, we have that $x+y \in F$
		\item Closure under multiplication: $\forall x, y \in F$, we have that $xy \in F$
	\end{enumerate}

\section{Random Matrix Theory}
	Random matrix theory is the study of matrices that arise with some probability distribution. This has interesting intersections with concentration inequality theory and obtaining bounds on eigenvalues for statistical estimates.

	We say that the vector space of hermitian matrices is $H_n(K)$ with field $K \in \{\mathbb{R}, \mathbb{C}\}$. The general vector space of all $n \times n$ matrices is $M_n(K)$.

\section{Resolvent Operator}
	
	In its most general form, we can start off by defining the Cauchy-Stieljes transform for all $z \in \mathbb{C}_+ = \{z \in \mathbb{C}: \eta(z) > 0 \}$:

		$$g_\mu(z) = \int \frac{1}{\lambda - z}  d\mu(\lambda)$$

	where $\mu$ is a finite measure on $\mathbb{R}$. 

	% defining the resolvent
	If we consider a matrix, $A \in H_n(\mathbb{C})$ and $\psi \in \mathbb{C}^n$ is a vectro with unit l2-norm. The spectral theorem guarantees the existence of $(v_1, ..., v_n)$ set of orthonormal basis vectors that are \textbf{eigenvectors} of A, such that:

		$$Av_i = \lambda_i v_i$$

	We say that the \textbf{spectral measure with vector $\psi$} is the real probability measure defined by:

		$$\mu_A^\psi = \sum_{k=1}^n |\langle v_k,  \psi \rangle |^2 \delta_{\lambda_k}(A)$$

	If $z \notin \sigma(A)$, then A - zI is invertible. We define the resolvent of A as the function: $R: \mathbb{C}_+ \mapsto M_n(\mathbb{C})$
		
		$$R(z) = (A - zI)^{-1}$$

	We derive basic properties of the resolvent matrix:

	\begin{lemma} [Basic Properties of the Resolvent Matrix]
		Let $A \in H_n(\mathbb{C})$ and R(z) is the resolvent. For any $z \in \mathbb{C}_+$ and $1 \le i$, and $j \le n$, we have that:

		\begin{enumerate}
			\item Analytic: $z \mapsto R(z)_{ij}$ is an analytic function
			\item Bounded: $||R(z)|| \le |z|^{-1}$
			\item Normal
		\end{enumerate}
	\end{lemma}


\end{document}